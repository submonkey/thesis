\noindent
\changeSize{\JudulIndonesia}

\noindent 
\begin{table}
    \centering
    \begin{tabular}{l l l}
        Nama Mahasiswa&: & \penulis \\
        NRP&: & \nrp \\
        Pembimbing &: & \pembimbingSatu \\
        %Pembimbing II&: & \pembimbingDua \\
    \end{tabular}
\end{table}

\changeSize{ABSTRAK}

Perkembangan teknologi informasi dan komunikasi mempengaruhi cara orang
berinteraksi dengan obyek yang terkait dengan dirinya. Salah satu proses
interaksi yang dibutuhkan dalam lingkungan bergerak adalah proses \tracking.
Secara umum, proses \tracking~adalah proses mengamati orang atau benda yang
bergerak secara kontinyu dimana obyek-obyek yang diamati terus dimonitor baik
posisi maupun aktifitasnya. Proses \tracking~yang ideal dapat mengirimkan
perubahan lokasi secara terus-menerus dalam kondisi yang berubah-ubah. Namun
demikian sistem \tracking~seperti ini umumnya kurang efisien karena dapat
menghabiskan \f{resource} baik daya maupun kebutuhan \f{bandwidth} jaringan
sehingga membutuhkan proses yang lebih efisien.

Sistem \tracking~tradisional kurang efisien untuk dikembangkan menjadi
infrastruktur \tracking~multi target pada perangkat bergerak dimana baik
pengamat ataupun obyek yang diamati lebih dari satu. Dibutuhkan mekanisme
komunikasi yang \f{loosely coupled}. \PubSub~memiliki kelebihan
\decoupling~pada waktu, ruang dan sinkronisasi. Kelebihan interaksi seperti
ini, menjadikan sistem \pubsub~ideal dalam komunikasi skala besar yang dinamis.
Efisiensi lain dapat dilakukan dengan melakukan \tracking~secara adaptif dan
bersifat \f{context awareness}. Pembaruan lokasi berdasarkan pada kondisi
\tracking~yang diberikan oleh pengamat serta kondisi objek pengamatan secara
adaptif.

\noindent \\ 
\bo{Kata kunci}: \Tracking~Multi Target, \PubSub, \f{Context Awareness} 

\cleardoublepage
