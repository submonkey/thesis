\chapter{\babSatu}
\section{Latar Belakang}

Perkembangan teknologi informasi dan komunikasi yang cepat, terutama dalam
internet dan telepon seluler telah mengubah cara orang berinteraksi dengan
lingkungannya.  Telepon seluler secara revolusioner memungkinkan orang untuk
berkomunikasi kapan saja dan dimana saja. Adanya hubungan ini menyebabkan perlu
adanya konsep berbagi informasi.  Informasi-informasi ini dapat berupa data
profil pribadi, berkas, lokasi, suhu lingkungan yang didapatkan melalui sensor
dan lain-lain.

Fitur perangkat bergerak sudah dilengkapi dengan kemampuan untuk penentuan
lokasi memanfaatkan GPS (\GPS).  Fitur ini tidak hanya digunakan secara lokal
pada perangkat tersebut, namun dapat juga dimanfaatkan secara \f{remote}
seperti halnya dalam \f{GPS tracking}. Agar penggunaan menjadi bermanfaat,
proses \tracking~haruslah ramah energi untuk mencegah konsumsi energi berlebih
pada perangkat bergerak. Optimasi efisiensi pada perangkat bergerak menjadi isu
yang hangat dengan berbagai macam pendekatan. Seperti halnya optimasi
menurunkan konsumsi energi dalam protokol jaringan serta optimasi pada level
sistem operasi. 

Suatu proses \tracking~dikatakan ideal, apabila dapat mengirimkan perubahan
lokasi secara \f{robust} dalam kondisi yang berubah-ubah. Beberapa penelitian
yang ada (Civilis dkk, 2005) (Leonhardi dkk, 2001) (Leonhardi dkk, 2002) telah
mengusulkan adanya \tracking~secara dinamis untuk memperbaharui lokasi.
\tracking~secara dinamis meminimalkan beban pada server dengan menurunkan
jumlah pembaharuan, tetapi masih mempunyai nilai akurasi yang relevan.
Leonhardi dkk (Leonhardi dkk, 2001) mempelajari \tracking~berbasiskan waktu
serta jarak yang membutuhkan akurasi pada benda bergerak. Simulasi dilakukan
dengan melakukan sejumlah pembaharuan pada beberapa teknik \tracking~untuk
mengukur akurasinya. 

EnTracked (Kjaergaard dkk, 2010), adalah sebuah sistem \tracking~yang efisien
pada lingkungan perangkat bergerak untuk target tunggal. Sistem ini dapat
mendeteksi kapan diperlukan pembaharuan lokasi memanfaatkan GPS atau tidak
secara adaptif. Deteksi pergerakan memanfaatkan sensor \acc~yang tertanam pada
perangkat bergerak. Sensor GPS digunakan pada kondisi bergerak atau telah
melewati batas waktu yang ditentukan melalui \f{error model}. Selebihnya lokasi
akan diestimasi menggunakan estimasi kecepatan pergerakan. Hasil dari
\tracking~ini akan disimpan pada \f{EnTracked Server}, untuk kemudian
dikirimkan ke klien yang membutuhkan.

Mekanisme seperti ini, kurang efisien untuk menjadi infrastruktur
\tracking~multi target pada perangkat bergerak. Komunikasi \f{point-to-point}
dan \f{synchronous} menyebabkan aplikasi menjadi kaku dan statis, serta membuat
pengembangan skala besar yang dinamis menjadi rumit (Eugster dkk, 2003). Untuk
mengurangi beban pada aplikasi, diperlukan skema komunikasi yang bersifat
\f{loosely coupled}. \PubSub~memiliki kelebihan \decoupling~pada waktu, ruang
dan sinkronisasi. Interaksi seperti ini, menjadikan sistem \pubsub~ideal dalam
komunikasi skala besar yang dinamis.

\PubSub~merupakan sebuah paradigma interaksi dimana \subscriber~mempunyai suatu
ketertarikan berupa suatu \event, untuk setiap \event~yang cocok dengan
ketertarikan akan dinotifikasi oleh \f{publisher} (Eugster dkk, 2003). Banyak
skema untuk menentukan suatu \event~yang menjadi ketertarikan \subscriber. Dua
skema yang banyak digunakan adalah \topicbased~serta \contentbased. Skema
\topicbased~bersifat statis dan primitif, dapat diimplementasikan secara
efisien. Lain halnya dengan \contentbased~yang bersifat ekspresif, tetapi
sebagai konsekuensinya dibutuhkan suatu komponen \f{filtering} yang mempunyai
kompleksitas yang tinggi.

Adaptasi \pubsub~pada lingkungan perangkat bergerak dipaparkan oleh Huang dkk
(Huang dkk, 2004). Dalam penelitiannya dipaparkan beberapa arsitektur
\pubsub~dari yang paling sederhana hingga terdistribusi. Sistem
\f{publish-subscribe} dapat secara dinamis mengirimkan \event~kepada
\subscriber. Dibutuhkan \f{event queues} serta \f{broker handoff protocol}
untuk membangun konsep \pubsub~dalam lingkungan perangkat bergerak. Penelitian
ini mendasari \f{Mobile XSiena}, suatu \f{platform} yang merupakan pengembangan
\f{XSiena} \pubsub~berbasis konten (Salvador dkk, 2010).

Salah satu fitur utama dari \f{publish-subscribe} dapat melakukan
\f{content-filtering}. Dengan adanya ini pengguna hanya mendapatkan
pembaharuan yang diinginkan sesuai dengan ketertarikan. Beban pada sisi klien
dapat berkurang, karena klien tidak perlu melakukan mekanisme \f{polling} ke
server pada waktu tertentu. Pada sistem \pubsub~tradisional, \publisher~akan
tetap mengirimkan \event~kepada server, walaupun tidak ada
\subscription~terhadap \event~tersebut.

Penentuan metode pengambilan lokasi pada \tracking, mempengaruhi dalam
efisiensi energi.  Metode dengan sensor GPS, memberikan nilai presisi lebih
tinggi dari metode lain serta membutuhkan energi yang lebih tinggi juga. Oleh
karena itu dibutuhkan sistem \tracking~yang memperhatikan \context~\aware.
\Context~berisi informasi yang dapat digunakan untuk menggambarkan situasi dari
entitas. Entitas dapat berupa orang, tempat atau obyek yang dianggap relevan
untuk interaksi antara pengguna dan aplikasi (Hong dkk, 2009).  Pada sistem
EnTracked, penentuan lokasi menggunakan sensor GPS hanya digunakan ketika obyek
bergerak.

Dari uraian di atas, dapat diketahui bahwa kelebihan \pubsub~dapat dimanfaatkan
dalam proses pembaharuan \tracking~pada lingkungan skala besar yang dinamis.
Dengan adanya ini sisi client cukup mendaftarkan ketertarikan sebagai
\event~kepada server. Pengiriman informasi lokasi pada \tracking~secara
sekuensial, menyebabkan perlu adanya penentuan lokasi yang efisien tetapi tetap
akurat. Dengan adanya sistem yang bersifat \context~\aware, sistem dapat secara
adaptif menentukan kebutuhan tingkat presisi informasi. Karakteristik
\publisher~yang tetap mengirimkan \event~kepada server dapat membebani sistem,
terutama dalam proses \f{content-filtering}. Oleh karena itu, dalam penelitian
ini diajukan suatu metode untuk mengoptimalkan proses \tracking~dengan sifat
\context~\aware~pada arsitektur \pubsub.  Integrasi sistem \tracking~akan
dilakukan pada perangkat bergerak yang memiliki sensor GPS di dalamnya.

\section{Perumusan Masalah}
Permasalahan yang akan dibahas dalam penelitian ini, yaitu:
\begin{enumerate}[noitemsep, nolistsep]
    \item Bagaimana memodelkan skema \tracking~multi target dengan model komunikasi
        \pubsub~dalam lingkungan bergerak.
    \item Bagaimana menangani \f{query} \tracking~multi target dalam lingkungan
        perangkat bergerak yang bersifat \f{unreliable}.
    \item Bagaimana melakukan optimasi \f{update protocol} pada sistem 
        \tracking~multi target secara adaptif untuk efisiensi trafik komunikasi.
\end{enumerate}

\section{Batasan Penelitian}

Dalam penelitian ini, batasan masalah yang dibahas diuraikan sebagai berikut:
\begin{enumerate}[noitemsep, nolistsep]
    \item Walaupun masalah keamanan menjadi masalah utama dalam 
        komunikasi jaringan, hal ini tidak ditekankan pada penelitian ini.
    %\item Diasumsikan posisi \f{smartphone} tetap dan pada satu tempat yaitu berada 
        %di saku celana depan.
    \item Sistem \PubSub~menggunakan arsitektur tunggal (terpusat).
    \item Pemodelan lokasi \tracking~hanya dilakukan untuk wilayah kampus Insitut 
        Teknologi Sepuluh Nopember yang terbagi menjadi tiga level hirarki.
    %\item Area \tracking~dalam uji coba, hanya meliputi wilayah kampus Institut Teknologi
        %Sepuluh Nopember dan bersifat \f{outdoor}.
\end{enumerate}

\section{Tujuan dan Manfaat Penelitian}

%Tujuan penelitian ini adalah untuk menghasilkan mekanisme \tracking~multi target
%dengan model komunikasi \pubsub~dalam lingkungan bergerak secara adaptif. 
%Membangun sistem tracking multi target \pubsub 
Tujuan penelitian yang dapat dicapai dalam penelitian ini dapat diuraikan sebagai
berikut:
\begin{enumerate}[noitemsep, nolistsep]
    \item Membangun sistem \tracking~multi target berbasis \pubsub~dalam lingkungan
        bergerak.
    \item Membangun \f{update protocol} yang bersifat adaptif berdasarkan 
        \context~pengguna pada sistem \tracking~multi target.
    \item Mengimplementasikan dan menguji secara nyata sistem \tracking~multi 
        target dalam lingkungan bergerak.
\end{enumerate}

%membangun modul adapter adaptif update

%Mengimplementasikan secara nyata dalam lingkungan bergerak.

Sedangkan manfaat penelitian ini terciptanya sistem \tracking~multi target
secara lebih efisien pada lingkungan bergerak. Efisiensi dalam lingkungan
bergerak menjadi hal utama dikarenakan keterbatasan \f{resource} baik daya
maupun kebutuhan jaringan.

%Terciptanya sistem multi target secara lebih efisien.

\section{Kontribusi Penelitian}

Kontribusi penelitian ini adalah pembangunan suatu mekanisme interaksi yang
memfasilitasi \tracking~multi target untuk multi user dengan efisiensi pada
proses \f{update protocol} secara adaptif.
